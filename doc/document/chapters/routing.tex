One essential feature of our application is its capability to not only consider calendar items for itself but to put them in their geographical context and to account for local traffic facilities. This implies the need for a set of data representing these facilites and an engine to operate on the data.\newline

From a very abstract and general perspective, this set of data can be split into two categories, namely a dynamic and a static one. As static, in these terms, one should label all facilities which do not depend on any schedule and may in principle by used at any time (for example streets and highways). Dynamic facilities are the ones, that have a more or less fixed schedule constraining their usage.\newline

In this project we will only consider static facilities, since also accounting for dynamic ones will yield to a much more complex application structure. Besides this aspect, it is a big issue to collect data about dynamic traffic facilities.\newline
 
As potential sources for static traffic facility data, one finds basically two options

\begin{itemize}
 
	\item \textbf{Google Maps}
	
	\item \textbf{Openstreetmap}
	
		Openstreetmap is an open database for a world wide map. Everyone is free to add, change and remove map items. Consequently its map material varies over a hugh scale in both quality and actuality. 
  
\end{itemize} 

For our project, Openstreetmap seems to be the one to choose.
