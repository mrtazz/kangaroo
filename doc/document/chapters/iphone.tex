The iPhone is a very interesting device for mobile applications,
since it has a very modern and intuitive interface. However due
to some limitations in the iPhone OS, it is not suitable for
mobile task routing and planning.

The main limitation of the iPhone OS is the fact that it does
not allow applications to run in the background. This was a
design decision to increase battery life. This is a big problem,
since the main functionalities of the application rely on the fact
that it is always running. For example, if the application isn't
running all the time, it is not possible to warn the user about
several situations in which he needs to act to fulfill his schedule.

In the iPhone OS there are two solutions possible to circumvent this
limitation. The first one would be to have the application running all
the time, which is both impracticle and would mean that nothing else
can be done with the device during that. The second solution, which was
integrated in the newest major release of the iPhone OS, is a
notification system, which makes it possible to push messages,
notification sounds and application badges onto the device. This
however would mean, that the routing engine needs to be implemented
on a server or other external device. Then we could get position updates
from the iPhone, calculate if there was anything to be done and
notify the user accordingly. However such a scenario would impose
more threats and problems on personal privacy and is therefore not
wanted. Additionally it is not completely sure, that the position data
of the device, used for the ``Find my iPhone'' functionality, for
example, is publicly available for developers.

That is why the iPhone is not suitable as a platform.
