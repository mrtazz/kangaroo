All data integration on the Android plattform is done via
so called ``content providers''. These providers are the
only way to transfer data between applications, since there
is no shared space which all applications could use.

Content providers provide a database like interface to the content to be
accessed. This means for every access a query has to be created and executed.
This query object, has to be compiled from several Android specific strings and
content URIs and given an SQL like query to determine the content to be fetched
from the provider.

\subsubsection{Calendar Content Provider API} % (fold)
\label{ssub:Calendar Content Provider API}

Unfortunately according to the Android development ressources, the content
provider for calendar data is non-existent. However after thorough research, it
became clear that the calendar data API exists, but is neither documented nor
officially released. This means all the access specifics have to be reverse
engineered and can also change at any time. Additionally there is no definite
statement on when the API will be released or what it will look like.

Since this
is a serious problem for the stability and usability of the application, it has
to be considered thoroughly in the design process.

% subsubsection Calendar Content Provider API (end)

\subsubsection{Calendar High Level Abstraction} % (fold)
\label{ssub:Calendar High Level Abstraction}


In order to not be too dependent on the changing API, an high level abstraction
layer was built around the calendar access. This enabled the application to
access calendar data as pure Java objects.
Additionally there are now only four
methods which have to be adapted on an API change, making the application
virtually independent from the content provider backend

\begin{itemize}
  \item queryEvents
  \item insertEventToBackend
  \item updateEventInBackend
  \item deleteEventFromBackend
\end{itemize}

The ``queryEvents'' method can be given a SQL statement as a selection
statement and a String array. Values in the array subsequently replace all
the ``?'' in the statement. The method then returns an ArrayList of
CalendarEvent objects, which suited the selection.

The ``insertEventToBackend'' method is used to insert new events via the
content provider. It takes a CalendarEvent object, builds an Android
ContentValues object from it and inserts this into the backend.

The ``updateEventInBackend'' method does practically the same, but updates
the events in the backend, instead.

At last the ``deleteEventFromBackend'' method removes the passed CalendarEvent
completely from the backend.

These four methods, described above, are the sole fundament of the calendar
wrapper library. As they mostly consist of less than ten lines of code,
the adaption to a new API can be done extremely easily.

% subsubsection Calendar High Level Abstraction (end)
